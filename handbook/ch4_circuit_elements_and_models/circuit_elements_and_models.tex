\chapter{Circuit Elements and Models} \label{chapter:circuit_elements_and_models}

\section{Basic lines}

\subsection{.TITLE line}
\textnormal{Examples:}
\begin{lstlisting}
	Circuit based on ATmega8A
	* additional lines following
	*...

	****************************
	* additional lines following
	*...
	.TITLE Circuit based on ATmega328p
	* additional lines following
	*...
\end{lstlisting}
\textnormal{The title line can be the first line or a
\lstinline|.TITLE <any title>| placed anywhere in the input file.}

\subsection{.END line}
\textnormal{Examples:}
\begin{lstlisting}
	.end
\end{lstlisting}
\textnormal{The \lstinline|.end| line must always be the last in the input
file. Note that the period is the integral part of the name.}

\subsection{Comments}
\textnormal{General Form:}
\begin{lstlisting}
	* <any comment>
\end{lstlisting}
\textnormal{Examples:}
\begin{lstlisting}
	* OLED two-wire should be connected to PD4,PD5 of XU1
	* Type of PWM can be selected using TCCR1A and
	* TCCR1B registers
\end{lstlisting}
\textnormal{The asterisk in the first column indicates that this line is
a comment line. Comment lines can be placed anywhere in the circuit
description.}

\subsection{End-of-line comments}
\textnormal{General Form:}
\begin{lstlisting}
	<any command> $ <any comment>
	<any command> //<any comment>
\end{lstlisting}
\textnormal{Examples:}
\begin{lstlisting}
	XQ1 3(1) 3(5) 0(8) 0(16) m8a $ Connect power and ground
	.MODEL m8a ATMEGA8A_32A (freq=2meg) //Run at 2MHz
\end{lstlisting}
\textnormal{MCUSim supports comments that begin with double marks
'\$ '(dollar plus space) or '//'. You should precede each comment character
with a space for readability. Simulator will also accept a comment line
started with the double marks '\$ '.}

\section{.MODEL Device Models}
\textnormal{General Form:}
\begin{lstlisting}
	.model mname type(pname1=pval1 pname2=pval2 ... )
\end{lstlisting}
\textnormal{Examples:}
\begin{lstlisting}
	.model m8a ATMEGA8A_32A(freq=2meg)
	.model m328p ATMEGA328P_32A(freq=8meg efuse=0x3)
\end{lstlisting}
\textnormal{MCUSim has been created to provide complex device models to
simulate microcontrollers. The models usually have a lot of parameters
which can be adjusted. It is highly likely that the default values of
the parameters won't let you run a simulation correctly. For this reason,
a set of the device model parameters is defined in a separate
\lstinline|.model line| and assigned a unique \lstinline|model name|.}

\section{.SCR}
\textnormal{General Form:}
\begin{lstlisting}
	.scr filename sname
\end{lstlisting}
\textnormal{Examples:}
\begin{lstlisting}
	.scr /users/mcusim/common/dht11.lua DHT11
	.scr /users/mcusim/common/ssd1309.lua OLED12864
\end{lstlisting}
\textnormal{The \lstinline|.SCR| statement allows you to load a specifically
structured script into MCUSim and use a device model defined there. Please,
note that the model can be referenced using a unique \lstinline|sname|,
configured and used the very similar way as the device models embedded into
the simulator.}
